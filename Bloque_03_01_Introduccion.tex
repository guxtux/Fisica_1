\documentclass[14pt]{beamer}
\usepackage{./Estilos/BeamerUVM}
\usepackage{./Estilos/ColoresLatex}
\input{./Preambulos/preambulo_Beamer_Warsaw_seahorse}
% \usefonttheme{serif}
\usepackage[clock]{ifsym}

\sisetup{per-mode=symbol}
\resetcounteronoverlays{saveenumi}

\title{\Large{Bloque 3} \\ \normalsize{Física I}}
\date{12 de mayo de 2023}

\begin{document}
\maketitle

\section*{Contenido}
\frame{\frametitle{Contenido} \tableofcontents[currentsection, hideallsubsections]}

\section{Mecánica clásica}
\frame{\frametitle{Temas} \tableofcontents[currentsection, hideothersubsections]}
\subsection{Fuerzas y aceleración}

\subsection{Leyes de Newton}

\subsection{Ley de Gravitación Universal}

\section{Trabajo y energía}
\frame{\frametitle{Temas} \tableofcontents[currentsection, hideothersubsections]}
\subsection{Trabajo mecánico}

\subsection{Energía cinética}

\subsection{Energía potencial}

\subsection{Conservación de energía}


\begin{frame}
\frametitle{Mecánica clásica}

\end{frame}

\end{document}