\documentclass[14pt]{extarticle}
\usepackage[utf8]{inputenc}
\usepackage[T1]{fontenc}
\usepackage[spanish,es-lcroman]{babel}
\usepackage{amsmath}
\usepackage{amsthm}
\usepackage{physics}
\usepackage{tikz}
\usepackage{float}
\usepackage[autostyle,spanish=mexican]{csquotes}
\usepackage[per-mode=symbol]{siunitx}
\usepackage{gensymb}
\usepackage{multicol}
\usepackage{enumitem}
\usepackage[left=2.00cm, right=2.00cm, top=2.00cm, 
     bottom=2.00cm]{geometry}

%\renewcommand{\questionlabel}{\thequestion)}
\decimalpoint
\sisetup{bracket-numbers = false}

\title{\vspace*{-2cm} Ejercicios de Evaluación Continua \\ Curso de Física 1\vspace{-5ex}}
\date{\today}
\begin{document}
\maketitle

\textbf{Indicaciones:} Resuelve de manera detallada cada uno de los siguientes ejercicios, en donde deberás de manejar en cada paso las correspondientes unidades. El puntaje de cada ejercicio es de $\mathbf{0.25}$ \textbf{puntos}. Si el ejercicio presenta solo el resultado con las unidades, aportará solo la mitad del puntaje.

\begin{enumerate}
\item Calcula la fuerza de atracción gravitacional entre la Tierra y la Luna, sabiendo que sus masas son \hfill \SI{5.98d24}{\kilo\gram} y \SI{7.35d22}{\kilo\gram}, respectivamente, y están separados una distancia de \SI{3.8d8}{\meter}.
\item Calcula la distancia promerdio entre la Tierra y el Sol, cuyas masas son \SI{5.98d24}{\kilo\gram} y \SI{2d30}{\kilo\gram}, respectivamente, si entre ellos existe una fuerza de atracción gravitacional de \SI{3.6d22}{\newton}.
\item Un cuerpo de \SI{60}{\kilo\gram} se encuentra a una distancia de \SI{3.5}{\meter} de otro cuerpo, de manera que entre ellos se produce una fuerza de \SI{6.5d-7}{\newton}. Calcula la masa del otro cuerpo.
\item Un planeta tiene dos lunas de igual masa. La luna $1$ está en una órbita circular de radio $r$. La luna $2$ está en órbita de radio $2 r$. ¿Cuál es la magnitud de la fuerza gravitacional que ejerce el planeta sobre la luna $2$?

a) Cuatro veces mayor que sobre la luna $1$.

b) Dos veces mayor que sobre la luna $1$.

c) Igual que sobre la luna $1$.

d) La mitad de la ejercida sobre la luna $1$.

e) Un cuarto de la ejercida sobre la luna $1$.

En este último ejercicio no necesitarás la calculadora para responder, solo plantea debidamente tu solución, tendrás que apoyarte un poco en el álgebra. 
\end{enumerate}

\end{document}