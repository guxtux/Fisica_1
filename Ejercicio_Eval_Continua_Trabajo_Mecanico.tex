\documentclass[14pt]{extarticle}
\usepackage[utf8]{inputenc}
\usepackage[T1]{fontenc}
\usepackage[spanish,es-lcroman]{babel}
\usepackage{amsmath}
\usepackage{amsthm}
\usepackage{physics}
\usepackage{tikz}
\usepackage{float}
\usepackage[autostyle,spanish=mexican]{csquotes}
\usepackage[per-mode=symbol]{siunitx}
\usepackage{gensymb}
\usepackage{multicol}
\usepackage{enumitem}
\usepackage[left=2.00cm, right=2.00cm, top=2.00cm, 
     bottom=2.00cm]{geometry}

%\renewcommand{\questionlabel}{\thequestion)}
\decimalpoint
\sisetup{bracket-numbers = false}

\title{\vspace*{-2cm} Ejercicios de Trabajo mecánico \\ Curso de Física 1\vspace{-5ex}}
\date{\today}
\begin{document}
\maketitle

\textbf{Indicaciones:} Resuelve de manera detallada cada uno de los siguientes ejercicios, en donde deberás de manejar en cada paso las correspondientes unidades. El puntaje de cada ejercicio es de $\mathbf{1}$ \textbf{punto}. Si el ejercicio presenta solo el resultado con las unidades, aportará solo la mitad del puntaje.

\begin{enumerate}
\item Para arrancar un auto de \SI{800}{\kilo\gram} de transmisión estándar \enquote{en segunda}, se necesita como mínimo recorrer \SI{5}{\meter} de distancia. Si entre varios amigos realizaron un trabajo de \SI{3.5d4}{\joule} al empujarlo:
\begin{enumerate}
\item ¿Lograron que el auto arrancara?
\item ¿Qué trabajo se requiere exactamente para que el auto arranque?
\end{enumerate}
\item ¿Qué fuerza necesita aplicar una grúa para subir $8$ niveles de \SI{2.5}{\meter} cada uno, si desarrolla un trabajo de \SI{1.25d5}{\joule}
\end{enumerate}

\end{document}