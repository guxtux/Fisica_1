\documentclass[14pt]{beamer}
\usepackage{./Estilos/BeamerUVM}
\usepackage{./Estilos/ColoresLatex}
\usetheme{Madrid}
\usecolortheme{default}
%\useoutertheme{default}
\setbeamercovered{invisible}
% or whatever (possibly just delete it)
\setbeamertemplate{section in toc}[sections numbered]
\setbeamertemplate{subsection in toc}[subsections numbered]
\setbeamertemplate{subsection in toc}{\leavevmode\leftskip=3.2em\rlap{\hskip-2em\inserttocsectionnumber.\inserttocsubsectionnumber}\inserttocsubsection\par}
% \setbeamercolor{section in toc}{fg=blue}
% \setbeamercolor{subsection in toc}{fg=blue}
% \setbeamercolor{frametitle}{fg=blue}
\setbeamertemplate{caption}[numbered]

\setbeamertemplate{footline}
\beamertemplatenavigationsymbolsempty
\setbeamertemplate{headline}{}


\makeatletter
% \setbeamercolor{section in foot}{bg=gray!30, fg=black!90!orange}
% \setbeamercolor{subsection in foot}{bg=blue!30}
% \setbeamercolor{date in foot}{bg=black}
\setbeamertemplate{footline}
{
  \leavevmode%
  \hbox{%
  \begin{beamercolorbox}[wd=.333333\paperwidth,ht=2.25ex,dp=1ex,center]{section in foot}%
    \usebeamerfont{section in foot} {\insertsection}
  \end{beamercolorbox}%
  \begin{beamercolorbox}[wd=.333333\paperwidth,ht=2.25ex,dp=1ex,center]{subsection in foot}%
    \usebeamerfont{subsection in foot}  \insertsubsection
  \end{beamercolorbox}%
  \begin{beamercolorbox}[wd=.333333\paperwidth,ht=2.25ex,dp=1ex,right]{date in head/foot}%
    \usebeamerfont{date in head/foot} \insertshortdate{} \hspace*{2em}
    \insertframenumber{} / \inserttotalframenumber \hspace*{2ex} 
  \end{beamercolorbox}}%
  \vskip0pt%
}
\makeatother

\makeatletter
\patchcmd{\beamer@sectionintoc}{\vskip1.5em}{\vskip0.8em}{}{}
\makeatother

% \usefonttheme{serif}

\sisetup{per-mode=symbol}
\resetcounteronoverlays{saveenumi}

\title{\Large{Prácticas de Laboratorio} \\ \normalsize{Física I}}
\date{27 de abril de 2023}

\begin{document}
\maketitle

\section*{Contenido}
\frame[allowframebreaks]{\frametitle{Contenido} \tableofcontents[currentsection, hideallsubsections]}

\section{Reglamento}
\frame[allowframebreaks]{\frametitle{Temas a revisar}\tableofcontents[currentsection, hideothersubsections]}
\subsection{Uso del Laboratorio}

\begin{frame}
\frametitle{Capítulo II}
\textocolor{carmine}{Artículo 12.} \pause Se prohíbe la introducción de alimentos y bebidas al laboratorio.
\\
\bigskip
\pause
\textocolor{carmine}{Artículo 13.} \pause Dentro del laboratorio se prohibe fumar o \textocolor{red}{prender fuego no autorizado} para realizar las prácticas correspondientes.
\end{frame}

\subsection{Obligaciones}

\begin{frame}
\frametitle{Capítulo IV}
\textocolor{ao}{Artículo 20.} Los estudiantes tendrán las siguientes obligaciones:
\pause
\setbeamercolor{item projected}{bg=black,fg=white}
\setbeamertemplate{enumerate items}{%
\usebeamercolor[bg]{item projected}%
\raisebox{1.5pt}{\colorbox{bg}{\color{fg}\footnotesize\insertenumlabel}}%
}
\begin{enumerate}[<+->]
\item Cumplir con las normas de higiene y seguridad establecidas para el laboratorio.
\item Presentarse puntualmente con el material requerido a la práctica a realizar.
\item Observar buena conducta dentro del laboratorio.
\seti
\end{enumerate}
\end{frame}
\begin{frame}
\frametitle{Capítulo IV}
\setbeamercolor{item projected}{bg=black,fg=white}
\setbeamertemplate{enumerate items}{%
\usebeamercolor[bg]{item projected}%
\raisebox{1.5pt}{\colorbox{bg}{\color{fg}\footnotesize\insertenumlabel}}%
}
\begin{enumerate}[<+->]
\conti
\item Informar al profesor de los desperfectos que detecte en el uso de los equipos e instalaciones.
\item Entregar los reportes necesarios de cada práctica conforme lo señale el manual correspondiente elaborado por la academia.
\item Evitar el uso de teléfonos celulares y cualquier dispositivo electrónico de uso personal dentro de los laboratorios.
\end{enumerate}
\end{frame}

\section{Normas de Seguridad}
\frame{\frametitle{Temas a revisar}\tableofcontents[currentsection, hideothersubsections]}
\subsection{Indicaciones}

\begin{frame}
\frametitle{Puntos de cumplimiento}
\setbeamercolor{item projected}{bg=bananayellow,fg=ao}
\setbeamertemplate{enumerate items}{%
\usebeamercolor[bg]{item projected}%
\raisebox{1.5pt}{\colorbox{bg}{\color{fg}\footnotesize\insertenumlabel}}%
}
\begin{enumerate}[<+->]
\item No jugar o hacer bromas con los compañeros dentro del laboratorio.
\item En la mesa de trabajo debe estar únicamente el material y las sustancias con las cuáles se va a experimentar, así como el manual de prácticas de cada uno.
\seti
\end{enumerate}
\end{frame}
\begin{frame}
\frametitle{Puntos de cumplimiento}
\setbeamercolor{item projected}{bg=bananayellow,fg=ao}
\setbeamertemplate{enumerate items}{%
\usebeamercolor[bg]{item projected}%
\raisebox{1.5pt}{\colorbox{bg}{\color{fg}\footnotesize\insertenumlabel}}%
}
\begin{enumerate}[<+->]
\conti
\item No se debe comer o beber en el laboratorio, recuerda todas las sustancias que se encuentran dentro del laboratorio son reactivos.
\item No se debe fumar o encender cerillos sin autorización.
\end{enumerate}
\end{frame}

\subsection{Sanciones}

\begin{frame}
\frametitle{En caso de incumplimiento}
El reglamento es muy claro con respecto al comportamiento dentro del Laboratorio, \pause en caso de que no se sigan los lineamientos, el Profesor podrá cancelar la práctica al equipo o a un alumno, reportando a la Coordinación la eventualidad.
\end{frame}
\begin{frame}
\frametitle{Prácticas no realizadas}
Una práctica que no se concluya o no se presente por indisciplina, no podrá reponerse posteriormente, \pause pero formará parte del esquema de evaluación.
\end{frame}


\section{Previo a la práctica}
\frame{\frametitle{Temas a revisar}\tableofcontents[currentsection, hideothersubsections]}
\subsection{Trabajo en el Manual}

\begin{frame}
\frametitle{Revisión inicial}
Cada práctica dentro del Manual incluye una sección:
\pause
\begin{center}
\textocolor{auburn}{Investiga y Escribe Brevemente}
\end{center}
\pause
Que el alumno deberá de reponder previo a la clase. \pause Antes de la sesión en Laboratorio, el Profesor indicará el número de Práctica a realizar, por lo que \textocolor{cobalt}{CADA ALUMNO} responderá en su guía, la sección señalada.
\end{frame}

\subsection{Objetivos}

\begin{frame}
\frametitle{Redacción de los objetivos}
Cada alumno deberá de redactar en el apartado \textocolor{darkmagenta}{OBJETIVOS}, lo que considere como meta de la Práctica.
\\
\bigskip
\pause
Así mismo, deberá de consultar \textocolor{black}{Fuentes Bibliográficas} para incluirlas en la sección correspondiente (pueden ser libros, revistas, enlaces de sitios web, etc.)
\end{frame}

\section{La práctica}
\frame{\tableofcontents[currentsection, hideothersubsections]}
\subsection{Los equipos}

\begin{frame}
\frametitle{Integrantes del equipo}
Un equipo de trabajo deberá de formarse con un máximo de $4$ alumnos.
\\
\bigskip
\pause
Favoreciendo la participación de cada integrante en cada una de las actividades que se deban de desarrollar.
\end{frame}
\begin{frame}
\frametitle{De la conformación de los equipos}
El Profesor puede elegir a los integrantes de cada equipo.
\\
\bigskip
\pause
En caso de una negativa para formar parte del equipo, el alumno no participará en la práctica, notificando a la Coordinación.
\end{frame}
\begin{frame}
\frametitle{Trabajo en equipo}
Se entiende como trabajo en equipo, el compromiso de cada integrante en participar de manera activa en la práctica.
\\
\bigskip
\pause
Si un alumno se desatiende de la práctica, no se le tomará en cuenta el reporte de la misma.
\end{frame}

\section{Conclusiones}
\frame[allowframebreaks]{\frametitle{Temas a revisar}\tableofcontents[currentsection, hideothersubsections]}
\subsection{En el Manual}

\begin{frame}
\frametitle{Escribiendo unas primeras Conclusiones}
Cada Práctica incluye un apartado de \textocolor{darkgreen}{Conclusiones}, \pause cada alumno deberá de responder en la guía las preguntas que se indican.
\end{frame}
\begin{frame}
\frametitle{Revisión de las conclusiones}
Cuando el Profesor solicite el Manual de Prácticas a cada alumno, revisará las respuestas en el apartado de Conclusiones, contabilizando como parte de la Evaluación Continua.
\end{frame}
\begin{frame}
\frametitle{Respuestas esperadas}
Se espera que cada respuesta en las Conclusiones sea un enunciado elaborado y apoyado con los resultados de la práctica.
\end{frame}
\begin{frame}
\frametitle{Respuestas esperadas}
Respuestas tipo: \pause \enquote{Si se alcanzaron}, \pause \enquote{Si es importante para mi formación académica}, \pause no son respuestas con una justificación, por lo que deberán de extender más su enunciado.
\end{frame}

\section{Reporte de Práctica}
\frame[allowframebreaks]{\frametitle{Temas a revisar}\tableofcontents[currentsection, hideothersubsections]}
\subsection{Entrega individual}

\begin{frame}
\frametitle{Reporte personal}
Si bien el trabajo de la Práctica es en equipo, \pause la entrega del reporte será \textocolor{cadetblue!80!black}{INDIVIDUAL}.
\end{frame}

\subsection{Elementos del Reporte}

\begin{frame}
\frametitle{Objetivos}
El objetivo de la Práctica que se redactó en el Manual es el que se deberá de incluir en el Reporte.
\end{frame}
\begin{frame}
\frametitle{Material}
Se debe de enlista el material que se ocupó en la Práctica.
\end{frame}
\begin{frame}
\frametitle{Desarrollo}
Se deberá de redactar el procedimiento que realizaron para el trabajo en la Práctica, \pause así como la mención de incidencias que se sucitaron y cómo se resolvieron. 
\end{frame}
\begin{frame}
\frametitle{Resultados}
Las mediciones, registros y anotaciones que se elaboraron, se deberán de incluir de manera organizada, preferentemente en tablas.
\end{frame}
\begin{frame}
\frametitle{Gráficas}
El incluir gráficas de los datos organizados, es un punto que apoyará bastante al Reporte.
\\
\bigskip
\pause
Será fundamental para la redacción de las Conclusiones.
\end{frame}
\begin{frame}
\frametitle{Conclusiones}
Las Conclusiones que se anotaron en el Manual, \pause deberán de estar en este apartado, de manera extendida, justificando cada una de ellas.
\end{frame}
\begin{frame}
\frametitle{Formato de entrega}
La entrega individual del Reporte se hará de manera física, \pause ya sea ocupando un procesador de textos o escrito a mano.
\end{frame}
\begin{frame}
\frametitle{Fecha de entrega}
Una vez concluida la práctica (en jueves), el Reporte se entregará a la siguiente clase del curso (en martes).
\\
\bigskip
\pause
No se recibirán Reportes de manera extemporánea.
\end{frame}

\section{Evaluación}
\frame[allowframebreaks]{\frametitle{Temas a revisar}\tableofcontents[currentsection, hideothersubsections]}
\subsection{Retroalimentación}

\begin{frame}
\frametitle{Uso de la rúbrica}
Para la evaluación del Reporte se utilizará la rúbrica contenida en cada una de las Prácticas del Manual.
\\
\bigskip
\pause
Por lo que cada alumno ya sabe los aspectos que se van a evaluar.
\end{frame}
\begin{frame}
\frametitle{Elementos de la evaluación}
Consideren que en la rúbrica hay tres primeros aspectos a evaluar:
\pause
\setbeamercolor{item projected}{bg=aquamarine,fg=black}
\setbeamertemplate{enumerate items}{%
\usebeamercolor[bg]{item projected}%
\raisebox{1.5pt}{\colorbox{bg}{\color{fg}\footnotesize\insertenumlabel}}%
}
\begin{enumerate}[<+->]
\item Preparación.
\item Trabajo experimental.
\item Participación.
\end{enumerate}
\end{frame}
\begin{frame}
\frametitle{Elementos de evaluación}
Estos aspectos se calificarán a criterio del Profesor, por lo que se les invita a mantener una participación constante, dinámica y concentrada durante la clase de Laboratorio.
\end{frame}
\begin{frame}
\frametitle{Elementos de evaluación}
Durante la sesión de Laboratorio, el Profesor registrará cualquier incumplimiento en los elementos de la rúbrica, anotando y firmado en el Manual del alumno que incumpla los puntos de Preparación, Trabajo Experimental y Participación.
\end{frame}

\end{document}